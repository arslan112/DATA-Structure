\documentclass[11pt]{article}            % Report class in 11 points
\parindent0pt  \parskip10pt             % make block paragraphs
\usepackage{graphicx}
\usepackage{listings}
\graphicspath{ {images/} }
\usepackage{graphicx} %  graphics header file
\begin{document}
\begin{titlepage}
    \centering
  \vfill
    \includegraphics[width=8cm]{uni_logo.png} \\ 
	\vskip2cm
    {\bfseries\Large
	Data Structure and Algorithms \\ 
	
	\vskip2cm
	Lab Report 
	 
	\vskip2cm
	}    

\begin{center}
\begin{tabular}{ l l  } 

Name: & Muhammad Arslan Ishaq \\ 
Registration \#: & CSU-F16-112 \\ 
Lab Report \#: & 06 \\ 
 Dated:& 21-05-2018\\ 
Submitted To:& Mr. Usman Ahmed\\ 

 %\hline
\end{tabular}
\end{center}
    \vfill
    The University of Lahore, Islamabad Campus\\
Department of Computer Science \& Information Technology
\end{titlepage}


    
    {\bfseries\Large
\centering
	Experiment \# 6 \\

Introduction to Doubly Linked list \\
		}    
 \vskip1cm
 \textbf {Objective}\\  The objectives of this lab session are to understand the basics of doubly linked list.\\
 \textbf {Software Tool} \\
1.   Dev\ C++


\section{Theory }              
A doubly-linked list is a linked data structure that consists of a set of sequentially linked records called nodes. Each node contains two fields, called links, that are references to the previous and to the next node in the sequence of nodes. Doubly linked list program in C++. Each Node will have a reference pointer to its next as well as previous node. \\  \\
\textbf{Advantages: :- }\\ 
1. We can traverse in both directions i.e. from starting to end and as well as from end to starting.
2. It is easy to reverse the linked list.
3. If we are at a node, then we can go to any node. But in linear linked list, it is not possible to reach the previous node. .  \\ \\
\textbf{Disadvantages:- }\\
1. It requires more space per space per node because one extra field is required for pointer to previous node.
2. Insertion and deletion take more time than linear linked list because more pointer operations are required than linear linked list. 

\textbf{Inserting and Deleting: - }\\
To insert a node before another, we change the link that pointed to the old node, using the prev link; then set the new node's next link to point to the old node, and change that node's prev link accordingly.. \\

As in doubly linked lists, "removeAfter" and "removeBefore" can be implemented with "remove(list, node.prev)" and "remove(list, node.next)". \\ \\
 
\section{Lab Task }  
Write a C++ code using functions for the following operations.
1. Create a Doubly Link List.
2. Traversing a Linked List\\ \\ 

\subsection{Program}     

\begin{lstlisting}[language=c++]
#include<iostream>
#include<stdlib.h>
#include<conio.h>
using namespace std;
struct node{
	int data;
	node* pre;
	node* next;
};

node* head = NULL;

node* getNode(int data){
	node* newNode = (node*)malloc(sizeof(node));
	(*newNode).data = data;
	(*newNode).pre = NULL;
	(*newNode).next = NULL;
	return newNode;
}

void insert(node* newNode){
	node* last_node = (node*)malloc(sizeof(node));
	last_node = head;
	head = newNode;
	newNode -> pre = NULL;
	newNode -> next = last_node;
	cout<<"\n\nData inserted successfully.\n\n";
	cout<<"\n\nPress any key to continue...............";
	getch();
	return;
}

void show(){
	node* newNode = (node*)malloc(sizeof(node));
	newNode = head;
	cout<<"\n\nData in the list\n\n";
	while(newNode != NULL){
		cout<<newNode -> data<<" ";
		newNode = newNode -> next;
	}
	cout<<"\n\nPress any key to continue..";
	getch();
	return;
}

int main(){
	int choice, data;
	node* newNode;
	up:
	system("cls");
	cout<<"\n\n\t Enter your Choice\n";
	cout<<"\t1. Insert Data\n";
	cout<<"\t2. Show Data\n";
	cout<<"\t3. Exit\n\n\t";
	cin>>choice;
	if(choice == 1){
		cout<<"\n\nEnter data to insert: ";
		cin>>data;
		newNode = getNode(data);
		insert(newNode);
		goto up;
	}
	else if(choice == 2){
		show();
		goto up;
	}
	else if(choice == 3){
		exit(0);
	}
	else{
		cout<<"\n\nWrong Input Choice!";
		cout<<"\n\nPress any key to continue........";
		getch();
		goto up;
	}
	return 0;
}

\end{lstlisting}

\begin{center}
 \includegraphics[width=10cm]{Capture.png}\\ 
\textbf{Figure : 1 Inserting Data}
\vskip 0.5cm
\end{center}

\begin{center}
 \includegraphics[width=10cm]{Capture1.png}\\ 
\textbf{Figure : 2 Displaying Data}
\vskip 0.5cm
\end{center}

\section{Conclusion}  
The insertion and remove operation works in the doubly linked list, adding extra features such as  add/removing items by index, smart traversal, etc. should be easier.It is important that you have a solid conceptual understand of these basic data structures before attempting to move onto some of the more complex data structures.I recommend playing around with the doubly linked list by adding the aforementioned features such as insertion/removal by index. In the source code, I have added these features for your reference.

 
\end{document}                          % The required last line
